\documentclass{article}
    % General document formatting
    \usepackage[margin=0.7in]{geometry}
    \usepackage[parfill]{parskip}
    \usepackage[utf8]{inputenc}
    % Related to math
    \usepackage{amsmath,amssymb,amsfonts,amsthm}
    \renewcommand{\baselinestretch}{1.5}
\title{Reportes Semanales}
\author{Andrea Sánchez}
\begin{document}
\maketitle
    \textbf{Semana del 27 al 31 de Enero 2020:}\newline 
    La meta principal para la primera semana de investigación fue preparar las herramientas computacionales para poder realizar las actividades plaenadas a futuro, tales como la utilización análisis estadístico para obtener información de datos empleando diferentes modelos matemáticos.
    El objetivo de esto fue presentar, estudiar y detallar cada paso del seguimiento que se escolta para realizar una tarea. Las herramientas computacionales distinguidas que ofrecen orden y estructura al equipo son Git\footnote{Sistema de control de versiones.} y Docker\footnote{Proyecto de código abierto que automatiza el despliegue de aplicaciones dentro de contenedores de software.}, que son necesarias durante el análisis de datos ya que segmenta el proceso en pequeños compenentes.

    Por parte de Docker elaboramos un contenedor donde es posible para el equipo trabajar bajo las mismas condiciones de sistema, incluso aunque cada miembro de éste labore con sistemas operativos diferentes. Por último, en Git obtenemos las intrucciones que debe seguir el contenedor y así represente al equipo una plataforma ideal para terminar la tarea. El objetivo de todo lo anterior es poder trabajar de manera interactiva con el equipo de Ciencia de Datos y reducir el tiempo en el ciclo de análisis para mejorar la calidad.


\begin{thebibliography}{1} 
\bibitem{Bergh C.,} Bergh, C., Benghiat, G., \& Strod, E. (2019). The Data OPS Cookbook.. Cambridge, MA, EUA: DataKitchen Headquarters.
\end{thebibliography}
\end{document}